
\documentclass[a4paper]{article}
\usepackage[OT1]{fontenc}
\usepackage{Sweave}
\usepackage{graphicx}
\usepackage{myVignette}
%\bibliographystyle{plainnat}
%\VignetteIndexEntry{An introduction to lifecontingencies package}
%\VignetteKeywords{vig1}
%\VignettePackage{lifecontingencies}
\begin{document}



\title{An introduction to lifecontingencies package}
\author{Giorgio A. Spedicato}

\maketitle

\section{Overview}
I've decided to submit to the cran package to perform life contingencies calculation in order to fill a current
lack within the CRAN archive. I will fill this  vignette further, nevertheless I anticipate the structure:

\begin{enumerate}
	\item Section \ref{sec:package} provides an overview of R usage within actuarial 
	fields and describes the package structure.
	\item Section \ref{sec:examples} gives a wide choice of lifecontingencies packages example.
	\item Finally section \label{sec:discussion} will provide a discussion of results and further potential
	developments.
\end{enumerate}

Please note these vignette are a work in progress.

\section{Lifecontingencies package description} \label{sec:package}

\subsection{Current R actuarial packages}
R \cite{rSoftware} represents a powerful enviromnent for statistical analysis and simulation. Thus many packages 
dedicated to P\&C actuarial software have been available from some years. Among those we shall cite:
\begin{itemize}
	\item The package \textbf{actuar} \cite{Dutang2008} provides functions to fit relevant loss distribution and perform credibility 
	analysis. It represents the computational side of the classical book \cite{lossDistributions}.
	\item The package \textbf{chainladder} \cite{chainLadder} provides functions to estimate non-life loss reserve.
\end{itemize}
The choice of statistical functions to perform rate - making is more wide as R provides a wide range of statistical function to 
perform classification and predictive modelling task (e.g. GLMs, data - mining techniques) performed by pricing actuaries.\\

Life actuaries works more with demographic and financial data. While R has a dedicated view to packages dedicated to financial analysis and 
few packages exists to perform demographic analysis (see for examples \cite{demography}) as of August 2011 no package exists to perform 
life contingencies calculation.\\
This is the reason I dediced to write functions to perform life contingencies calculation, even if my professional histories lies in the non-life 
side by far.\\


\subsection{The structure of the package}

The package contains R function, classes and methods to perform classical financial mathematics calculations, working with 
lifetable objects and classical life contingencies calculations.\\
Demos and vignettes (like this document) are also available.\\

The package is load within the R working environment as follows:

\begin{Schunk}
\begin{Sinput}
> library(lifecontingencies)
\end{Sinput}
\end{Schunk}



\clearpage
\newpage


\section{Examples} \label{sec:examples}

\subsection{Classical financial mathematics example}

Two examples will show classical financial mathematics applications of package lifecontingencies. The 
amortization of a loan and a savings account projection.\\

\subsubsection{Loan amortization}

\begin{Schunk}
\begin{Sinput}
> capital = 1e+05
> interest = 0.05
> payments_per_year = 2
> effectiveRate = (1 + interest)^(1/payments_per_year) - 1
> years = 10
> installment = capital/annuity(i = effectiveRate, periods = years * 
+     payments_per_year)
> installment
\end{Sinput}
\begin{Soutput}
[1] 6396.251
\end{Soutput}
\begin{Sinput}
> balance_due = numeric(years * payments_per_year)
> balance_due[1] = capital * (1 + effectiveRate) - installment
> for (i in 2:length(balance_due)) {
+     balance_due[i] = balance_due[i - 1] * (1 + effectiveRate) - 
+         installment
+     cat("Payment ", i, " balance due:", round(balance_due[i]), 
+         "\n")
+ }
\end{Sinput}
\begin{Soutput}
Payment  2  balance due: 92050 
Payment  3  balance due: 87926 
Payment  4  balance due: 83702 
Payment  5  balance due: 79372 
Payment  6  balance due: 74936 
Payment  7  balance due: 70390 
Payment  8  balance due: 65733 
Payment  9  balance due: 60960 
Payment  10  balance due: 56069 
Payment  11  balance due: 51057 
Payment  12  balance due: 45922 
Payment  13  balance due: 40659 
Payment  14  balance due: 35267 
Payment  15  balance due: 29742 
Payment  16  balance due: 24080 
Payment  17  balance due: 18279 
Payment  18  balance due: 12334 
Payment  19  balance due: 6242 
Payment  20  balance due: 0 
\end{Soutput}
\end{Schunk}

\subsubsection{Saving account projection}

\begin{Schunk}
\begin{Sinput}
> cumulatedSavings <- function(amount, rate, periods) {
+     service_charge = 1
+     service_fee = (0.01 * min(100, amount) + 0.005 * max(0, min(50, 
+         amount - 100)))
+     invested_amount = amount - service_charge - service_fee
+     out = invested_amount * accumulatedValue(interestRates = rate, 
+         periods = periods)
+     return(out)
+ }
> savings_sequence = seq(from = 50, to = 300, by = 10)
> periods = 30 * 12
> yearly_rate = 0.025
> montly_effective_rate = (1 + yearly_rate)^(1/12) - 1
> cumulated_value = sapply(savings_sequence, cumulatedSavings, 
+     montly_effective_rate, periods)
\end{Sinput}
\end{Schunk}

\subsection{Working with lifetable and actuarial table objects}

Lifetable objects represent the basic class designed to handle life table calculations needed to evaluate
life contingencies. Actuarialtable class inherits from lifetable class.\\
Both have been designed using the S4 class framework.
To build a lifetable class object three items are needed:
\begin{enumerate}
	\item The years sequence, that is an integer sequence $0,1,\ldots, \omega-1$. It shall starts from zero and going 
	to the $\omega-1$ age (the age $x$ as $p_x=0$).
	\item The $l_x$ vector, that is the number of subjects living at the beginning of age $x$.
	\item The name of the life table.
\end{enumerate}

\begin{Schunk}
\begin{Sinput}
> x_example = seq(from = 0, to = 9, by = 1)
> lx_example = c(1000, 950, 850, 700, 680, 600, 550, 400, 200, 
+     50)
> fakeLt = new("lifetable", x = x_example, lx = lx_example, name = "fake lifetable")
\end{Sinput}
\end{Schunk}

After a lifecontingencies table has been created, basic probability calculations may be performed.

\begin{Schunk}
\begin{Sinput}
> pxt(fakeLt, 2, 1)
\end{Sinput}
\begin{Soutput}
[1] 0.8235294
\end{Soutput}
\begin{Sinput}
> qxt(fakeLt, 3, 2)
\end{Sinput}
\begin{Soutput}
[1] 0.1428571
\end{Soutput}
\begin{Sinput}
> exn(fakeLt, 5, 2)
\end{Sinput}
\begin{Soutput}
[1] 1.583333
\end{Soutput}
\end{Schunk}

A print (or show - equivalent) method is also available, reporting the x, lx, px and ex in tabular form.

\begin{Schunk}
\begin{Sinput}
> print(fakeLt)
\end{Sinput}
\begin{Soutput}
Life table fake lifetable 

   x   lx        px       ex
1  0 1000 0.9500000 5.980000
2  1  950 0.8947368 5.242105
3  2  850 0.8235294 4.741176
4  3  700 0.9714286 4.542857
5  4  680 0.8823529 3.647059
6  5  600 0.9166667 3.000000
7  6  550 0.7272727 2.181818
8  7  400 0.5000000 1.625000
9  8  200 0.2500000 1.250000
10 9   50 0.0000000 1.000000
\end{Soutput}
\end{Schunk}

An actuarialtable class inherits from the lifecontingencies class, but contains and additional slot: the interest rate
slot.\\

\begin{Schunk}
\begin{Sinput}
> irate = 0.03
> fakeAct = new("actuarialtable", x = fakeLt@x, lx = fakeLt@lx, 
+     interest = irate, name = "fake actuarialtable")
\end{Sinput}
\end{Schunk}

\subsection{Classical actuarial mathematics examples}

\subsection{A thorough examples}

\clearpage
\newpage

\section{Discussion} \label{sec:discussion}

\subsection{Analysis of software capabilities}
\subsection{Prospective developments}


\bibliography{packagebiblio}

\end{document}
